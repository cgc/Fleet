\hypertarget{index_intro_sec}{}\section{Introduction}\label{index_intro_sec}
\hyperlink{class_fleet}{Fleet} is a C++ library for programming language of thought models. In these models, you will typically specify a grammar of primitive operations which can be composed to form complex hypotheses. These hypotheses are best thought of as programs in a mental programming language, and the job of learners is to observe data (typically inputs and outputs of programs) and infer the most likely program to have generated the outputs from the inputs. This is accomplished in \hyperlink{class_fleet}{Fleet} by using a fully-\/\+Bayesian setup, with a prior over programs typically defined thought a Probabilistic Context-\/\+Free \hyperlink{class_grammar}{Grammar} (P\+C\+FG) and a likelihood model that typically says that the output of a program is observed with some noise.

\hyperlink{class_fleet}{Fleet} is most similar to L\+O\+Tlib (\href{https://github.com/piantado/LOTlib3}{\tt https\+://github.\+com/piantado/\+L\+O\+Tlib3}) but is considerably faster. L\+O\+Tlib converts grammar productions into python expressions which are then evaled in python; this process is flexible and powerful, but very slow. \hyperlink{class_fleet}{Fleet} avoids this by implementing a lightweight stack-\/based virtual machine in which programs can be directly evaluated. This is especially advantageous when evaluating stochastic hypotheses (e.\+g. those using \hyperlink{_random_8h_ae295082303ce2024a3de1c53dc99568e}{flip()} or \hyperlink{_random_8h_ac2ea1cac6b4c8cad207512d19abe42d7}{sample()}) in which multiple execution paths must be evaluated. \hyperlink{class_fleet}{Fleet} stores these multiple execution traces of a single program in a priority queue (sorted by probability) and allows you to rapidly explore the space of execution traces.

\hyperlink{class_fleet}{Fleet} is structured to automatically create this virtual machine and a grammar for programs from just the type specification on primitives. The bulk of a \hyperlink{class_fleet}{Fleet} program is therefore in specifying the primitives that are used and a likelihood model that scores any potential program against the data.

To accomplish this, \hyperlink{class_fleet}{Fleet} makes heavy use of C++ template metaprogramming. It requires strongly-\/typed functions and requires you to specify the macro F\+L\+E\+E\+T\+\_\+\+G\+R\+A\+M\+M\+A\+R\+\_\+\+T\+Y\+P\+ES in order to tell its virtual machine what kinds of variables must be stored. In addition, \hyperlink{class_fleet}{Fleet} uses a std\+::tuple named P\+R\+I\+M\+I\+T\+I\+V\+ES in order to help define the grammar. This tuple consists of a collection of \hyperlink{struct_primitive}{Primitive} objects, essentially just lambda functions and weights). The input/output types of these primitives are automatically deduced from the lambdas (using templates) and the corresponding functions are added to the grammar. Note that the details of this mechanism may change in future versions in order to make it easier to add grammar types in other ways. In addition, \hyperlink{class_fleet}{Fleet} has a number of built-\/in operations, which do special things to the virtual machine (including \hyperlink{struct_builtin_1_1_flip}{Builtin\+::\+Flip}, which stores multiple execution traces; \hyperlink{struct_builtin_1_1_if}{Builtin\+::\+If} which uses short-\/circuit evaluation; \hyperlink{struct_builtin_1_1_recurse}{Builtin\+::\+Recurse}, which handles recursives hypotheses; and \hyperlink{struct_builtin_1_1_x}{Builtin\+::X} which provides the argument to the expression). These are not currently well documented but should be soon. $\ast$\hypertarget{index_install_sec}{}\section{Installation}\label{index_install_sec}
\hyperlink{class_fleet}{Fleet} is based on header-\/files, and requires no additional dependencies. Command line arguments are processed in C\+L11.\+hpp, which is included in src/dependencies/.

The easiest way to begin using \hyperlink{class_fleet}{Fleet} is to modify one of the examples. For simple rational-\/rules style inference, try Models/\+Rational\+Rules; for an example using stochastic operations, try Models/\+Formal\+Language\+Theory-\/\+Simple.

\hyperlink{class_fleet}{Fleet} is developed using G\+CC 9 (version $>$8 required).\hypertarget{index_install_sec}{}\section{Installation}\label{index_install_sec}

\begin{DoxyCode}
code goes here
\end{DoxyCode}
\hypertarget{index_install_sec}{}\section{Installation}\label{index_install_sec}
\hyperlink{class_fleet}{Fleet} provides a number of simple inference routines to use. These are all displayed in Models/\+Formal\+Language\+Theory-\/\+Simple.\hypertarget{index_step1}{}\subsection{Markov-\/\+Chain Monte-\/\+Carlo}\label{index_step1}
\hypertarget{index_step2}{}\subsection{Search (\+Monte-\/\+Carlo Tree Search)}\label{index_step2}
\hypertarget{index_step3}{}\subsection{Enumeration}\label{index_step3}
etc...\hypertarget{index_install_sec}{}\section{Installation}\label{index_install_sec}

\begin{DoxyItemize}
\item Sample things, store in \hyperlink{class_top_n}{TopN}, then evaluate... 
\end{DoxyItemize}